\chapter{Combinatorics}

\section*{1.1. Assigning seats}
\addcontentsline{toc}{section}{1.1. Assigning seats}
Six girls and four boys are to be assigned to ten seats in a row, with the stipulations
that a girl sits in the third seat and a boy sits in the eighth seat. How many arrangements
are possible?

\vspace{1em}

\begin{proof}
    We compute the number of ways that the boy on the eighth seat and the girl on the third seat can be chosen:
    \[
        \binom{6}{1}\binom{4}{1} = 24
    \]

    Now, we see how many ways the other kids can be placed on the seats:
    \[
        !8 = 40,320
    \]

    Our final result is:
    \[
        24 \cdot 40,320 = 967,680
\]
\end{proof}

\section*{1.2. Number of outcomes}
\addcontentsline{toc}{section}{1.2. Number of outcomes}
One person rolls two six-sided dice, and another person flips six two-sided coins.
Which setup has the larger number of possible outcomes, assuming that the order matters?

\vspace{1em}

\begin{proof}
    The two dice rolls have $6^2 = 36$ possible outcomes, while the coin flips have $2^6 = 64$ outcomes.
\end{proof}

\section*{1.3. Subtracting the repeats}
\addcontentsline{toc}{section}{1.3. Subtracting the repeats}
\begin{enumerate}[(a)]
    \item From Eq. (1.6) we know that the number of ordered sets of three people chosen from five people
        is $5 \cdot 4 \cdot 3 = 60$. Reproduce this result by starting with the naive answer of $5^3 = 125$
        ordered sets where repetitions are allowed, and then subtracting off the number of triplets that have
        repeated people.
    \item It's actually not much more difficult to solve this problem in the general case where triplets
        are chosen from N people, instead of five. Repeat part (a) for a general N.
\end{enumerate}

\vspace{1em}

\begin{proof}
    \hfill
    \begin{enumerate}[(a)]
        \item The number of triplets that have three repeating people is obviously $5$, while the number of 
            triplets with two repetitions is given by $5 \cdot 4 \cdot 2 + 5 \cdot 1 \cdot 4 = 60$. 
            Therefore, the number of ordered sets of three people chosen from five people is $125 - 60 = 65$.

        \item The number of triplets that have three repeating people will be $N$, while the number of triplets
            that have two repeating people will be $N(N - 1) \cdot 2 + N(N - 1) = 3N(N-1)$. As a result,
            the number of ordered sets of three people chosen from five people is given by
            $N^3 - 3N(N - 1) - N = N^3 - 3N^2 + 2N$.
    \end{enumerate}
\end{proof}

\section*{1.4. Subtracting the repeats, again}
\addcontentsline{toc}{section}{1.4. Subtracting the repeats, again}
Repeat the task of Problem 1.3(a), but now in the case where you pick quadruplets (instead of triplets)
from five people.

\vspace{1em}

\begin{proof}
    The number of ordered sets of 4 people chosen from five people is $5 \cdot 4 \cdot 3 \cdot 2 = 120$.

    To find the number of repeating quadruplets, we split them into 3 categories:
    \begin{enumerate}
        \item \textbf{4 repeating people} The repeating person can be chosen in 5 ways and there is only one way
            to order (AAAAA), so there are 5 such quadruples.

        \item \textbf{3 repeating people} The repeating person can be chosen in 5 ways, and then the 
            non-repeating person in 4 ways, thus the number of unorderd quadruples is given 
            by $4 \cdot 5 = 20$. Since there are 4 (AAAB, AABA, ABAA, BAAA) ways to choose the 
            order, we get $4 \cdot 20 = 80$ ordered quadruples.

        \item \textbf{2 repeating people} The repeating person can be chosen in 5 ways, and the other 2 people
            can be chosen in $\binom{4}{2} = 6$ ways. Therefore, we have $5 \cdot 30 = 50$ unordered quadruples.
            The sets can be ordered in 12 ways, so we have $30 \cdot 12 = 360$ orderd quadruples with 2 
            repeating people.

        \item \textbf{2 groups of repeating people} The repeated persons can be chosen in $\binom{5}{2} = 10$ 
            ways and can be ordered in 6 modes (AABB, ABAB, ABBA, BBAA, BABA, BAAB), therefore the number
            of such sorted quadruples is 60.

            Therefore, the number of ordered sets of four people from five people is 
            \[
                5^4 - 5 - 80 - 360 - 60 = 120
            \] 
    \end{enumerate}
\end{proof}

\vspace{1em}

\section*{1.6. Many ways to count}
\addcontentsline{toc}{section}{1.6. Many ways to count}
How many different orderings are there of the six letters: A, A, A, B, B, C? \\
How many different ways can you think of to answer this question?

\vspace{1em}

\begin{proof}
    We analyze the possible orderings of the letters. The positions of the 3 A letters can be chosen 
    in $\binom{6}{3} = 20$ ways, and then the positions of the two B letters can be chosen in
    $\binom{3}{2} = 3$ modes. Since the C letter has the last position, we find that the number of possible
    orderings is now $20 \cdot 3 = 60$.

    The same method can be applied for any ordering of letter-position assignments (e.g. we choose the 
    possible positions of B and then of A and C). We'll get different formulas, but the result will be the
    same. There are $3 \cdot 2 = 6$ such ways of computing the number of possible orderings.
\end{proof}

\section*{1.7. Committees with a president}
\addcontentsline{toc}{section}{1.7. Committees with a president}
Two students are given the following problem: From $N$ people, how many ways are there to choose a committee
of $n$ people, with one person chosen as the president? One student gives an answer of $n\binom{N}{n}$, while
the other student gives an answer of $N\binom{N-1}{n-1}$.

\begin{enumerate}[(a)]
    \item By writing out the binomial coefficients, show that the two answers are equal.
    \item Explain the (valid) reasoning that lead to these two (correct) answers.
\end{enumerate}

\vspace{1em}

\begin{proof}
    \hfill
    \begin{enumerate}[(a)]
        \item We simply rewrite the expressions to obtain the equality:
        \[
            n\binom{N}{n} = n\frac{N!}{n!(N-n)!} = \frac{N!}{(n-1)!(N-n)!} = N\frac{(N-1)!}{(n-1)!(N-n)!}
                          = N\binom{N-1}{n-1}
        \]
        \item The president can be chosen in $N$ ways and then the rest of the group can be chosen in 
            $\binom{N-1}{n-1}$ modes, giving us $N\binom{N-1}{n-1}$ ways to form the committee.

            Without thinking about the president, the committee can be chosen in $\binom{N}{n}$ ways. In 
            such a committee any of the $n$ persons can be chosen as the president. Therefore, we have 
            $n\binom{N}{n}$ ways to form the committee.
    \end{enumerate}
\end{proof}

\section*{1.8. Multinomial coefficients}
\addcontentsline{toc}{section}{1.8. Multinomial coefficients}
\begin{enumerate}[(a)]
    \item A group of ten people are divided into three committees. Three people are on committee
        A, two are on committee B, and five are on committee C. How many different ways are there 
        to divide up the people?

    \item
        A group of N people are divided into k committees. $n_1$ people are on committee 1, $n_2$
        people are on committee 2, \dots, and $n_k$ people are on committee $k$, with 
        $n_1 + n_2 + \dots + n_k = N$. How many different ways are there to divide up the people?
\end{enumerate}

\vspace{1em}

\begin{proof}
    \hfill
    \begin{enumerate}[(a)]
        \item The people on committee A can be chosen in $\binom{10}{3} = 120$ ways and then the people on
            committee B can be chosen in $\binom{7}{2} = 21$ ways. The remaining people will be on the C
            committee. As a result, the group of ten people can be divided in $120 \cdot 21 = 2,520$ ways.
        
        \item We assign people to committees like in the previous example. The 1st committee's members
            can be chosen in $\binom{N}{n_1}$ modes, the 2nd committee's composition can be chosen in 
            $\binom{N - n_1}{N_2}$ ways, and so on. We observe (and can prove using induction), that the
            number of ways the i-th ($2 \leq i < k$) committee can be assembled is given by the expression:
            \[
                M_i = \binom{N - n_1 - \ldots - n_{i - 1}}{n_i}
            \]
            Therefore, the number of ways the group of people can be divided is given by:
            \begin{align*}
                M = \binom{N}{n_1} \prod_{i = 2}^{k}M_i 
                    =& \binom{N}{n_1} \binom{N - n_1}{n_2} \ldots \binom{N - n_1 - \ldots - n_{k - 1}}{n_k} \\
                    =& \frac{N!}{n_1! (N - n_1)!} \cdot \frac{(N - n_1)!}{n_2! (N - n_1 - n_2)!} \cdot \ldots 
                    \cdot \frac{(N - n_1 - \ldots - n_{k - 1})!}{n_3! (N - n_1 - \ldots - n_{k - 1} - n_k)!} \\
                    =& \frac{N!}{n_1!n_2! \ldots n_k!}
            \end{align*}
    \end{enumerate}
\end{proof}

\section*{1.9. One heart and one 7}
\addcontentsline{toc}{section}{1.9. One heart and one 7}
How many different five-card poker hands contain exactly one heart and exactly one 7? (If the hand
contains the 7 of hearts, then this one card satisfies both requirements.)

\vspace{1em}

\begin{proof}
    There are $52 - 13 - 4 + 1 = 36$ cards that are neither a heart nor a 7. We consider two cases:
    \begin{enumerate}[(i)]
        \item \textbf{The 7 of hearts is in the hand}, so the other 4 cards in the hand
            can be any that are neither a heart nor a 7. As a result, we have 
            $\binom{36}{4}= 58,905$ such hands.

        \item \textbf{The 7 of hearts is not in the hand.} There are 12 cards that are hearts and 
            not a 7 and 3 cards that are 7 but not a heart. We choose one of each and the rest of 
            the cards can be any that are neither a heart nor a 7. Hence, we get
            $12 \cdot 3 \binom{36}{3} = 257,040$ such hands.
    \end{enumerate}
    In conclusion, there are $286,110 + 257,040 = 315,945$ five-card hands that contain exactly
    one heart and exactly one 7.
\end{proof}


\section*{1.14. Yahtzee}
\addcontentsline{toc}{section}{1.14. Yahtzee}
In the game of Yahtzee, five dice are rolled in a group, with the order not mattering.

\begin{enumerate}[(a)]
    \item Using Eq. (1.16), how many unordered rolls (sets) are possible?

    \item In the spirit of the examples at the beginning of Section 1.7. reproduce the
        result in part (a) by determining how many unordered rolls there are
        of each general type (for example, three of one number and two of another, etc.)

    \item In the spirit of the example at the end of Section 1.7., show that the total
        number of \emph{ordered} Yahtzee rolls is $6^5 = 7776$.
\end{enumerate}

\vspace{1em}

\begin{proof}
    \hfill
    \begin{enumerate}[(a)]
        \item The number of unordered rolls is given by 
            \[
                \binom{5 + (6 - 1)}{6 - 1} = \binom{10}{5} = 252
            \] 

        \item We split the rolls in 7 types:
            \begin{enumerate}[(1)]
                \item All five rolls are the same (e.g. 66666). There are 6 sets of this type.

                \item Four rolls have the same value and the other roll has another value (e.g. 66665).
                    There are $6 \cdot 5 = 30$ sets of this type.
                
                \item Three rolls have the same value and the other two rolls have other (different 2 by 2)
                    values (e.g. 66654). There are  $6 \binom{5}{2} = 6 \cdot 10 = 60$ sets of this type.

                \item Three rolls have the same value and the other two rolls have both another (same) value 
                    (e.g. 66655). There are $6 \cdot 5 = 30$ sets of this type.

                \item Two rolls have the same value and the other three rolls have different (2 by 2) values
                    (e.g. 66654). There are $6 \binom{5}{2} = 60$ such sets.

                \item Two rolls have the same value, two rolls have another (same value) and the 
                    remaining roll has another different value (e.g. 66554). 
                    There are $6 \binom{5}{2} = 60$ such sets of rolls.

                \item Each roll has a different value. (e.g. 65432). There are $\binom{6}{5} = 6$ such sets.
            \end{enumerate}

        By summing all the types, we get that the number of unordered rolls is 
        \[
            6 + 30 + 60 + 30 + 60 + 60 + 6 = 252
        \] 

    \item We want to see in how many ways can the 5 roll types be ordered:
        \begin{enumerate}[(1)]
            \item All rolls are the same, so the set can be ordered in one way. The number of 
                ordered sets with the same 5 rolls is 6.

            \item Four rolls have the same value and the other roll has another value.
                The set can be ordered in 5 ways (we consider the different roll on each
                possible position). Therefore, there are $5 \cdot 30 = 150$ ordered sets of this type.

            \item Three rolls have the same value and the other two have other (different 2 by 2) values.
                The positions of the repeated values can be chosen in $\binom{5}{3} = 10$ and then there 
                are two ways to order the other values, so there are $10 \cdot 2 = 20$ ways to order the set.
                As a result, there are $20 \cdot 60 = 1200$ ordered sets of this type.

            \item Three rolls have the same value and the other two rolls have both another (same) value.
                There are $\binom{5}{3} = 10$ ways the sets can be ordered, so we get $10 \cdot 30 = 300$
                such ordered sets.

            \item Two rolls have the same value and the other three have other (different 2 by 2) values.
                As before, we find that the set of values can be ordered in 
                $\binom{5}{2} \cdot 3 \cdot 2 = 60$ ways.
                Hence, there are $60 \cdot 60 = 3600$ such ordered sets.

            \item Two rolls have the same value, two rolls have another (same value) and the 
                remining roll has another differente value. There are 
                $\binom{5}{2}\binom{3}{2} = 10 \cdot 3 = 30$ ways to order this set, so we get 
                $30 \cdot 60 = 1800$ such ordered sets.

            \item All rolls have different values. The sets can be ordered in $5! = 120$ ways. As a result,
                there are $120 \cdot 6 = 720$ such ordered sets.
        \end{enumerate}

        In conclusion, we get that the number of $\emph{ordered}$ sets of Yahtzee rolls is:
        \[
            6 + 150 + 1200 + 300 + 3600 + 1800 + 720 = 7776 = 6^5
        \] 
    \end{enumerate}
\end{proof}

\section*{1.16. Pascal sum 2}
\addcontentsline{toc}{section}{1.16. Pascal sum 2}
At the end of Section 1.8.3, we demonstrated the relation 
$\binom{n}{k} = \binom{n - 1}{k - 1} + \binom{n - 1}{k}$
by using the argument involving committees. Repeat this reasoning,
but now in terms of:
\begin{enumerate}[(a)]
    \item coin flips,
    \item the $(a + b)^n$ binomial expansion.
\end{enumerate}

\vspace{1em}

\begin{proof}
    \hfill
    \begin{enumerate}[(a)]
        \item The number of ways we can get $k$ tails from $n$ coin flips is given by 
            $\binom{n}{k}$. If we single out the first flip, we get two cases:
            \begin{enumerate}[(1)]
                \item The flip was heads, so the number of ways to get tails $k$ times
                    from the rest $n - 1$ of the flips is $\binom{n - 1}{k}$

                \item The flip was tails, so the number of ways to get the remaining 
                    $k - 1$ tails flips from the other $n - 1$ rolls is given by $\binom{n - 1}{k - 1}$
            \end{enumerate}

        Therefore, the number of ways we get $k$ tails from $n$ coin flips can be split into the above
        two cases, so:
        \[
            \binom{n}{k} = \binom{n - 1}{k} + \binom{n - 1}{k - 1}
        \] 

    \item The coefficient of the term $a^{n - k}b^k$ from the expansion of $(a + b)^n$ is given by 
        $\binom{n}{k}$. If we single out the first $(a + b)$ factor, we have two possible situations:
        \begin{enumerate}
            \item The factor was used in getting a power of $b$ in $a^{n - k}b^k$, so there are 
                $\binom{n - 1}{k - 1}$ factors to use for the other $k - 1$ powers of $b$ and the $n - k$ 
                powers of $a$, since 
                $\binom{n - 1}{n - k} = \binom{n - 1}{k - 1}$.

            \item The factor wasn't used in getting a power of $b$ in $a^{n - k}b^k$, so there are 
                $\binom{n - 1}{k}$ factors to use for the other $k$ b powers and the other $n - k - 1$ 
                powers of a, since $\binom{n - 1}{n - k - 1} = \binom{n - 1}{k}$.
        \end{enumerate}

    Therefore, the coefficient of $a^{n-k}b^k$ is given by:
    \[
        \binom{n}{k} = \binom{n - 1}{k} + \binom{n - 1}{k - 1}
    \] 
    \end{enumerate}
\end{proof}

\section*{1.17. Pascal diagonal sum}
\addcontentsline{toc}{section}{1.17. Pascal diagonal sum}
\begin{enumerate}[(a)]
    \item If we pick an unordered committee of three people from five people (A, B, C, D, E),
        we can list the $\binom{5}{3} = 10$ possibilities as shown in Table 1.19.
        We have grouped them according to which letter comes first. (The order of letters
        doesn't matter, so we've written each triplet in increasing alphabetical order.)
        The columns in the table tell us that we can think of $10$ as equaling $6 + 3+ 1$.
        Explain why it makes sense to write this sum as  $\binom{4}{2} + \binom{3}{2} + \binom{2}{2}$.

        \begin{table}[h]
            \centering
            \begin{tabular}{ccc}
                A B C & & \\ 
                A B D & & \\
                A B E & & \\
                A C D & B C D & \\
                A C E & B C E & \\
                A D E & B D E & C D E \\
            \end{tabular}
            \caption*{\textbf{Table 1.19:} Unordered triplets chosen from five people.}
        \end{table}

    \item You can also see from Table 1.15 and 1.16 that, for example 
        $\binom{6}{3} = \binom{5}{2} + \binom{4}{2} + \binom{3}{2} + \binom{2}{2}$.
    More generally,
    \begin{equation*}\tag{1.29}
        \binom{n}{k} = \binom{n - 1}{k - 1} + \binom{n - 2}{k - 2} + \binom{n - 3}{k - 3} + \ldots + 
                        \binom{k}{k - 1} + \binom{k - 1}{k - 1}
    \end{equation*}

    In words: A given number (for example, $\binom{6}{3}$) in Pascal's triangle equals the sum
    of the numbers in the diagonal string that starts with the number that is above and to the 
    left of the given number ($\binom{5}{2}$ in this case) and then proceeds upward to the right.
    So the string contains $\binom{5}{2}, \binom{4}{2}, \binom{3}{2}$ and $\binom{2}{2}$
    in this case.

    Prove Eq.(1.29) by making repeated use of Eq.(1.22), which says that each number in Pascal's
    triangle is the sum of the two numbers abot it (or just the "1" above it, if it occurs at the end 
    of a line).
\end{enumerate}

\begin{proof}
    \hfill
    \begin{enumerate}[(a)]
        \item We split the committees into 4 categories:
            \begin{enumerate}[(1)]
                \item The set contains A, so there are $\binom{4}{2} = 6$ such sets.
                \item The set contains B and doesn't contain A, so there are $\binom{3}{2} = 3$ sets.
                \item The set contains C and doesn't contain A or B, so there is $\binom{2}{2} = 1$
                \item The set doesn't contain A, B or C. There are obviously no such sets, as we
                    need 3 elements.
           \end{enumerate}

        Since the reunion of those sets contains all the possible unordered committees of 3 members,
        we obtain that:
        \[
            \binom{4}{2} + \binom{3}{2} + \binom{2}{2} = 6 + 3 + 1 = 10 = \binom{5}{3}
        \] 
        
        \item We prove using induction that:
        \begin{equation*}\tag{1.29}
            \binom{n}{k} = \binom{n - 1}{k - 1} + \binom{n - 2}{k - 2} + \ldots + 
                            \binom{k}{k - 1} + \binom{k - 1}{k - 1}
                            = \sum_{i = k}^{n} \binom{i - 1}{k - 1}, \forall k \in \mathbb{N^*}, k \leq n
        \end{equation*}
        for all $n \in \mathbb{N}, n \geq 2$.

        The base case is obviously valid, since
        \begin{align*}
            \binom{2}{1} &= \binom{1}{0} + \binom{0}{0} = 1 + 1 = 2 \\
            \binom{2}{2} &= \binom{1}{1} + \binom{0}{1} = 1 + 0 = 1
        \end{align*}

        We assume that the relation holds for a fixed $m \in \mathbb{N}, m \geq 2$, so we have:
        \[
            \binom{m}{k} = \sum_{i = k}^{m}\binom{i - 1}{k - 1}, \forall k \in \mathbb{N^*}, k \leq n
        \]

        Using (1.22), we obtain that: 
        \[
            \binom{m + 1}{k} = \binom{m}{k} + \binom{m}{k - 1} 
                             = \sum_{i = k}^{m}\binom{i - 1}{k - 1} + \binom{m}{k - 1} 
                             = \sum_{i = k}^{m + 1}\binom{i - 1}{k - 1} 
        \] 

        Since (1.22) holds true for the base case and the $m$ case implies the validity of the $m + 1$ case,
        we proved using induction that (1.29) holds for all $n \in \mathbb{N}, n \geq 2$.
    \end{enumerate}
\end{proof}
