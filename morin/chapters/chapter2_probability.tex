\chapter*{2. Probability}

\section*{2.2. Rules for three events}
\begin{enumerate}[(a)]
    \item Consider three events, $A$, $B$, and $C$. 
        If they are all independent of each other,
        show that
        \begin{equation*}\tag{2.76}
            P(A \text{ and } B \text{ and } C) = P(A) \cdot P(B) \cdot P(C) 
        \end{equation*}

    \item If they are (possibly) dependent, show that 
        \begin{equation*}\tag{2.77}
            P(A \text{ and } B \text{ and } C) = P(A) \cdot P(B | A) \cdot P(C | A \text{ and } B) 
        \end{equation*}

    \item If they are all mutually exclusive, show that
        \begin{equation*}\tag{2.78}
            P(A \text{ or } B \text{ or } C) = P(A) + P(B) + P(C)
        \end{equation*}

    \item If they are (possibly) nonexclusive, show that
        \begin{align*}\tag{2.79}
            P(A \text{ or } B \text{ or } C) =& P(A) + P(B) + P(C) \\
                                          &- P(A \text{ and } B) - P(A \text{ and } C) - P(B \text{ and } C) \\
                                          &+ P(A \text{ and } B \text{ and } C)
        .\end{align*}
\end{enumerate}

\begin{proof}
    \hfill
    \begin{enumerate}[(a)]
        \item Using (2.9) we find that:
            \[
                P(A \text{ and } B \text{ and } C) = P(A) \cdot P(B \text{ and } C | A) 
                = P(A) \cdot P(B | A) \cdot P(C | A \text{ and } B)
            .\] 
            Because the events are independent, $P(B | A) = P(B)$ and $P(C | A \text { and } B) = P(C)$, so
            \begin{equation*}\tag{2.76}
                P(A \text{ and } B \text{ and } C) = P(A) \cdot P(B) \cdot P(C). 
            \end{equation*}

        \item Proved at (a).

        \item By (2.18), 
            \begin{align*}
                P(A \text{ and } B \text{ and } C) 
                    &= P(A) + P(B \text{ or } C) - P(A \text{ and } (B \text{ or } C)) \\
                    &= P(A) + P(B) + P(C) - P(B \text{ and } C) - P(A) \cdot P(B \text{ or } C | A)
            .\end{align*}

            We compute the subtracted member:
            \begin{align*}
                P(A) \cdot P(B \text{ or } C | A) 
                    &= P(A) \cdot (P(B|A) + P(C|A) - P(B \text{ and } C | A))) \\
                    &= P(A \text{ and } B) + P(A \text{ and } C) - P(A \text{ and } B \text{ and } C)
            .\end{align*}

            By substituting in the initial expression, we get:
            \begin{align*}\tag{2.79}
                P(A \text{ or } B \text{ or } C) =& P(A) + P(B) + P(C) \\
                                              &- P(A \text{ and } B) - P(A \text{ and } C) - 
                                                P(B \text{ and } C) \\
                                              &+ P(A \text{ and } B \text{ and } C)
            .\end{align*}

            Since the events are mutually exclusive, all $\emph{and}$ probabilities are 0, so:
            \begin{equation*}\tag{2.78}
                P(A \text{ or } B \text{ or } C) = P(A) + P(B) + P(C)
            \end{equation*}

        \item Proved at (c).
    \end{enumerate}
\end{proof}

\section*{2.7. Proofreading}
Two people each proofread the same book. One person finds 100 errors, and the other finds 60. 
There are 20 errors common to both people. Assume that all errors are equally likely to be found (which is 
undoubtedly not true in practice), and also that the discovery of an error by a person is independent
of the discovery of that error by the other person. Given these assumptions, roughly how many error 
does the book have? $\emph{Hint:}$ Draw the picture similar to Fig. 2.1, and then find the probability 
of each person finding a given error.

\vspace{1em}

\begin{proof}
    \hfill

    Let $P(A)$ be the probability that a random error is one of the errors discovered 
    by the person with 100 errors, and respectively let $P(B)$ be the same for the person with 60 errors.
    Also, let $N$ be the estimated number of total errors.
    Since the discoveries of errors are independent events, 
    \[
        P(A \text{ and } B) = P(A) \cdot P(B) = \frac{100}{N} \cdot \frac{60}{N} = \frac{6000}{N^2}
    .\] 
    But we know that 20 errors are common between the two persons, so
    \[
        P(A \text{ and } B) = \frac{20}{N}
    .\] 
    Therefore, we get the estimated total number of errors:
    \[
        \frac{20}{N} = \frac{6000}{N^2} \iff N = 300
    .\] 
\end{proof}

\section*{2.9. Sock pairs}
\begin{enumerate}[(a)]
    \item Four red socks and four blue socks are in a drawer. You reach in and pull out two socks at random.
        What is the probability that you obtain a matching pair?

    \item Answer the same question, but now in the general case with $n$ red socks and $n$ blue socks.

    \item Presumably you answered the above questions by counting the relevant pairs of socks.
        Can you think of a quick probability argument, requiring no counting, that gives the
        answer to part (b) (and part(a))?
\end{enumerate}

\vspace{1em}

\begin{proof}
    \hfill
    \begin{enumerate}[(a)]
        \item The total number of possible extracted pairs is $\binom{8}{2} = 28$ and the number
            of possible matching pair (of any color) extractions is $2 \cdot \binom{4}{2} = 12$. 
            Therefore, the probability of pulling a matching pair is:
            
            \[
                \frac{12}{28} = \frac{3}{7} \approx 0.428
            \] 

        \item As before, the total number of possible extracted pairs is $\binom{2n}{2} = n(2n - 1)$ and
            the number of possible matching pair (of any color) extractions is 
            $2 \cdot \binom{n}{2} = n(n - 1)$. As a result, the probability of pulling a 
            matching pair is:
            \[
                \frac{n(n-1)}{n(2n - 1)} = \frac{n - 1}{2n - 1} \to \frac{1}{2}
            \] 
        \item The first sock can either be red or blue. After the extraction, the drawer contains
            $2n - 1$ socks and $n - 1$ socks of the matching color, so the probability of pulling
            a matching pair is given by:
            \[
                \frac{n - 1}{2n - 1} \to \frac{1}{2}
            \] 
    \end{enumerate}
\end{proof}

\section*{2.11 At least one 6}
Three dice are rolled. What is the probability of obtaining at least one 6? We solved
this in Section 2.3.1, but your task here is to solve it the long way, by adding up the
probabilities of obtaining exactly one, two, or three 6's.

\begin{proof}
    Since each dice can take values from 1 to 6, the dices can be rolled in $6^3 = 216$ ways. 
    Then, there are $3 \cdot 5 \cdot 5 = 75$ ways in which only one dice is a 6 (we assume
    each individual dice rolls a 6 and the other two roll differently), so the probability 
    of doing that is $\frac{75}{216}$. Also, there are $3 \cdot 5 = 15$  
    ways of having exactly two 6 dices, so the probability of this event is $\frac{15}{216}$. 
    Finally, the event of having all dices being rolled as sixes can occur in only one way, so the 
    probability of it happening is $\frac{1}{216}$. In conclusion, the probability of rolling at least a 6 
    from three dice rolls is given by:
    \[
        \frac{75}{216} + \frac{15}{216} + \frac{1}{216} = \frac{91}{216} \approx 0.421
    \] 
\end{proof}

\section*{2.15 My birthday}
\begin{enumerate}[(a)]
    \item You are in a room with 100 other people. Let $p$ be the probability that at least one
        of these 100 people has your birthday. Without doing any calculations, state whether $p$ 
        is larger, smaller, or equal to 100/365.

    \item Now calculate the exact value of $p$.
\end{enumerate}

\begin{proof}
    \hfill
    \begin{enumerate}[(a)]
        \item Assuming all birthdays are equally likely to be encountered and that a year has 365 days,
            the probability that at least one person of the 100 has the same birthday as me is strictly less
            than $\frac{100}{365}$, since some people may have the same birthday and only unique
            birthdays are counted. The $\frac{100}{365}$ probability would be acquired if we consider 
            the 100 people in the room as having unique birthdays.

        \item The probability that a random person doesn't have the same birthday as me is $\frac{364}{365}$,
            so the probability that none of the 100 people in the room has the same birthday as me is
            $(\frac{365}{365})^{100}$. Therefore, the probability that at least one of them has the 
            same birthday as me is:
            \[
                p = 1 - \bigg(\frac{364}{365}\bigg)^{100} \approx 0.24
            \] 
    \end{enumerate}
\end{proof}

\section*{2.16. My birthday, again}
We saw at the end of Section 2.4.1 that 253 is the answer to the question, "How many people (in addition to
me) need to be present in order for there to be at least a 1/2 chance that someone else has $\emph{my}$ 
birthday?" We solved this by finding the smallest $n$ for which $(364/365)^n$ is less than 1/2. Answer
this question again, by making use of the approximation in Eq. (7.14) in Appendix C. What is the answer
in the general case where there are N days in a year instead of 365? Assume N is large.

\vspace{1em}

\begin{proof}
    We are given the approximation formula:
    \begin{equation*}\tag{7.14}
        (1 + a)^n \approx e^{na}
    \end{equation*}

    We consider two cases:
    \begin{enumerate}[(1)]
        \item A year has 365 days. We've seen in the previous exercise that the probability that at 
            least one of $n$ given people has the same birthday as me is given by the expression:
            \[
                p_n = 1 - \bigg(\frac{364}{365}\bigg)^n 
            \] 

            Let's assume that $p_n \approx \frac{1}{2}$, then:
            \[
                \bigg(\frac{364}{365}\bigg)^n \approx \frac{1}{2} \iff
                \bigg(1 - \frac{1}{365}\bigg)^n \approx \frac{1}{2} 
            \] 

            Using (7.14), we have that:
            \[
               e^{-\frac{n}{365}} \approx \frac{1}{2}
            \] 

            By taking the logarithm of both sides and then negating the terms, we see that 
            $\frac{n}{365} \approx \ln{2}$, so the number of people that should be present such
            that there is a least $\frac{1}{2}$ chance that someone else has my birthday is:
            \[
                n = 365 \ln{2} \approx 253 
            \] 

        \item A year has $N$ days. Similarly to the previous exercise, we see that the probability
            of a person not having the same birthday as me is $\frac{N - 1}{N}$. Then it is 
            easily deduced that the probability of a person having the same birthday as me is:
            \[
                p_n = 1 - \bigg(\frac{N - 1}{N}\bigg)^n 
            \] 

            Let's assume that $p_n \approx \frac{1}{2}$, then:
            \[
                \bigg(\frac{N - 1}{N}\bigg)^n \approx \frac{1}{2} \iff
                \bigg(1 - \frac{1}{N}\bigg)^n \approx \frac{1}{2} 
            \] 

            Using (7.14), we have that:
            \[
               e^{-\frac{n}{N}} \approx \frac{1}{2}
            \] 

            By taking the logarithm of both sides and then negating the terms, we see that 
            $\frac{n}{N} \approx \ln{2}$, so the number of people that should be present such
            that there is a least $\frac{1}{2}$ chance that someone else has my birthday is:
            \[
                n \approx N\ln{2} \approx 0.693N
            \]
    \end{enumerate}
\end{proof}

\section*{2.18. A random game-show host}
Consider the following variation of the Game-Show Problem we discussed in Section 2.4.2. A game-show
host offers you the choice of three doors. Behind one of these doors is the grand prize, and behind
the other two are goats. The host announces that after you select a door (without opening it), he will
$\emph{randomly}$ open one of the other doors, and the result happens to be a goat. He then offers
you the chance to switch your choice to the remaining door. Should you switch or not? Or does it 
not matter?

\vspace{1em}

\begin{proof}
    Since the doors can be reordered and not change the setup of the problem, we can pick the first
    door without loss of generality. There are three equally likely possibilities
    for what is behind the three doors: PGG, GPG, and GGP, where P denotes the prize and G denotes
    a goat. Let us use the subscript $H$ to show that a door was opened by the host. Considering that
    the host cannot choose our door (the first one), we have the following door layouts after the host 
    opens a door:

    \pagebreak

    \begin{table}[h]
        \centering
        \begin{tabular}{cc}
            PG_H G & PGG_H \\ 
            GP_HG & GPG_H \\
            GG_HP & GGP_H
        \end{tabular}
    \end{table}

    Since the host doesn't choose the door with the prize, those layouts have 0 probability of 
    occuring. The encounters of the other 4 layouts are equally likely, so they have a probability 
    of occuring of $\frac{1}{4}$. We can see that if we keep the initial choice of the door,
    there is a $\frac{1}{2}$ possibility of winning. Likewise, if we switch the door, there 
    is a $\frac{1}{2}$ probability of winning. Therefore, it does not matter if we switch our choice or not.
\end{proof}

\section*{2.19. Boy girl problem with general information}
This problem is an extension of the Boy/Girl problem from Section 2.4.4. You should study that 
problem thoroughly before tackling this one. As in the original version of the problem, assume
that all processes are completely random. The new variation is the following:

You bump into a random person on the street who says, "I have two children. At least one of them
is a boy whose birthday is in the summer." What is the probability that the other child is also 
a boy? What if the clause is changed to, "whose birthday is on August 11th"? Or "who was born
during a particular minute on August 11th"? Or more generally, "who has a particular characteristic
that occurs with probability $p$ "? $\emph{Hint:}$ Make a table of all of the various possibilities,
analogous to the tables in Section 2.4.4.

\vspace{1em}

\begin{proof}
    Without taking into account the particular characteristic, we have the following possible
    children couples:
    
    \begin{table}[h]
        \centering
        \begin{tabular}{c c c c}
            BB & BG & GB & GG
        \end{tabular}
    \end{table}

    , where B represents a boy and G a girl. Each of these groups is equally likely and
    is encountered with a probability of $\frac{1}{4}$. If we add a subscript C to the children
    that possess the particular characteristic, we get the following possible pairs:

    \begin{table}[h]
        \centering
        \begin{tabular}{c c c c}
            B_CB_C & B_CG_C & G_CB_C & G_CG_C \\
            B_CB & B_CG & G_CB & G_CB \\
            BB_C & BG_C & GB_C & GG_C \\
            BB & BG & BB & GG
        \end{tabular}
    \end{table}

    Since the probability of a kid to have the characteristic is $p$, then the probability
    of not having it is $1 - p$. We split the groups in 3 categories (following the 
    lines in the table):
    \begin{itemize}
        \item Line 1: Both children have the characteristic. The probability that one
            given couple is of this type is $\frac{1}{4}p^2$
        \item Lines 2 and 3: Only one kid has the characteristic. The probability that such
            a group is encountered is $\frac{1}{4}p(1 - p)$
        \item Line 4: None of the kids has the characteristic. The probability that a given
            group is in this category is $\frac{1}{4}(1 - p)^2$
    \end{itemize}

    \pagebreak

    Now, we find that the groups that contain at least one boy who has the characteristic are:

    \begin{table}[h]
        \centering
        \begin{tabular}{c c c}
            B_CB_C & B_CG_C & G_CB_C \\
            B_CB & B_CG & \\
            BB_C & & GB_C & \\
        \end{tabular}
    \end{table}

    We have 3 groups that contain two boys and at least one of them has the characteristic
    (first column). Also, there are a total of 7 groups containing at least a boy with the 
    characteristic (3 groups from the first line, 2 from the second
    line and 2 from the third line). Therefore, knowing that one children is a boy that possesses
    the characteristic $p$, the probability that the other kid is also a boy is:
    \[
        P_{BB} = \frac{\frac{1}{4}p^2 + 2 \frac{1}{4}p(1 - p)}{3 \frac{1}{4}p^2 + 4 \frac{1}{4} p(1 - p)}
        = \frac{p^2 + 2p(1 - p)}{3p^2 + 4p(1 - p)}
        = \frac{2p - p^2}{4p - p^2} 
        = \frac{2 - p}{4 - p}
    \] 

    In the base case, the characteristic is "having a birthday in the summer", so $p = \frac{1}{4}$.
    We get that the probability of the other kid being a boy is:
    \[
        P_{BB} = \frac{2 - \frac{1}{4}}{4 - \frac{1}{4}} = \frac{7}{15} \approx 0.467
    \] 

    The second characteristic is "having a birthday on August 11th", so $p = \frac{1}{365}$.
    The sought probability is then
    \[
        P_{BB} = \frac{2 - \frac{1}{365}}{4 - \frac{1}{365}} = \frac{729}{1459} \approx \frac{1}{2}
    \] 

    Finally, if the characteristic is "being born during a particular minute on August 11th", 
    so $p = \frac{1}{365} \frac{1}{1440} = \frac{1}{525600}$
    the probability that the other children is a boy is:
    \[
        P_{BB} = \frac{2 - \frac{1}{525600}}{4 - \frac{1}{525600}} = \frac{1051199}{2102399} \approx \frac{1}{2}
    \] 
\end{proof}

\section*{2.20. A second test}
Consider the setup in the "False positives" example in Section 2.5. If we instead perform 
$\emph{two}$ successive tests on each person, what is the probability that a person
who tests positive both times actually has the disease?

\begin{proof}
    The setup provided in the "False positives" example is the following:
    \begin{itemize}
        \item 2\% of the overall population has the disease.

        \item If a person $\emph{does}$ have the disease, then the test has a 95\% chance of correctly
            indicating that the person has it. (So 5\% of the time, the test incorrectly indicates
            that the person doesn't have the disease.)

        \item If a person $\emph{does not}$ have the disease, then the test has a 10\% chance of
            incorrectly indicating that the person has it; this is a "false positive" result.
            (So 90\% of the time, the test correctly indicates that the person doesn't have the 
            disease.)
    \end{itemize}

    Let $N$ be the population number and let's consider two cases:
    \begin{enumerate}
        \item The person has the disease, and the tests were positive results.
            The probability of a positive result for a diseased person is 90\%.
            Therefore the probability of a diseased person being diagnosticated twice
            as positive is $95\% \cdot 95\% = 90.25\%$. 2\% of the population is diseased,
            so the number of diseased persons that are tested twice as positive is 
            $2\%N \cdot 90.25\% = 1.805\%N$.

        \item The person does not have the disease, and both tests are "false positives".
            The probability of a "false positive" for a healthy person is 10\%.
            As a result, the probability of a healthy person being tested twice as false positive
            is $10\% \cdot 10\% = 1\%$. 98\% of the population is healthy, so the number
            of healthy people that are tested twice as "false positives" is 
            $98\%N \cdot 1\% = 0.98\%N$.
    \end{enumerate}

    Therefore, the probability of a person who tests positive both times actually has the disease
    is:
    \[
        \frac{1.805\%N}{1.805\%N + 0.98\%N} = \frac{1.805\%N}{2.785\%N} \approx 0.6481 = 64.81\%
    \]
\end{proof}
