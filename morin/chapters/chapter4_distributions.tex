\chapter{Distributions}

\section*{4.2. Expectation of a continuous distribution}
\addcontentsline{toc}{section}{4.2. Expectation of a continuous distribution}
The expectation value of a $\emph{discrete}$ random variable is 
given in Eq. (3.4). Given a $\emph{continuous}$ random variable with
probability density  $\rho(x)$, explain why the expectation value is
given by the integral $\int{xp(x)}dx$.

\vspace{1em}

\begin{proof}
    The expectation value of a $\emph{discrete}$ is given by:
    \begin{equation*}\tag{3.4}
        E[X] = \sum_{i = 1}^{|\Omega|}p(x_i) x_i
    \end{equation*}

    Since we are talking about a $\emph{continuous}$ variable, 
    the set of outcomes is infinite. Therefore, let 
    \[
        P = \{[x_0, x_1], [x_1, x_2], \ldots, [x_{n - 1}, x_n]\}
    \]
    be a partition of the outcome set $\Omega$, where $x_0 < x_1 < x_2 < \ldots < x_n$. Now, let
    $\displaystyle x_i^* = x_i, \linebreak \Delta x_i = x_i - x_{i-1}$ for all $i$ and 
    let $\epsilon_x$ and $\epsilon_y$ be the smallest real numbers such that for 
    all $x \in [x_{i - 1}, x_i]$, $|x_i^* - x| < \epsilon_x$ and $|p(x_i^*) - p(x)| < \epsilon_p$, 

    We can see that as $|\Delta x_i| \to 0$, then  $\epsilon_x \to 0$ and $\epsilon_p \to 0$ too.
    Therefore, we partition the outcome set in a huge number of subsets
    such that for all $x \in [x_{i - 1}, x_i]$, we have that $x \to x_i^*$
    and $p(x) \to p(x_i^*)$. Then, the restriction of the density function 
    on such an interval is given by:
    \[
        \rho(x) = \rho(x_i^*) = \lim_{|\Delta x_i| \to 0} \frac{p(x_i^*)}{\Delta x_i}, 
        \forall x \in [x_{i - 1}, x_i]
    \] 

    The expected value of the function is now given by the sum:
    \[
        E[X] = \lim_{|\Delta x_i| \to 0}\sum_{i = 1}^{n} x_i^* p(x_i^*)
        = \lim_{|\Delta x_i| \to 0}\sum_{i = 1}^{n} x_i^* \rho(x_i^*) \Delta x_i
    \] 

    But this is a left Riemann sum, so
    \[
        E[X] = \lim_{|\Delta x_i| \to 0}\sum_{i = 1}^{n} x_i^* \rho(x_i^*) \Delta x_i
        = \int x\rho(x) dx
    \] 
\end{proof}

\section*{4.3. Variance of the uniform distribution}
\addcontentsline{toc}{section}{4.3. Variance of the uniform distribution}
Using the general idea from Problem 4.2, find the variance of a uniform
distribution that extends from $x = 0$ to $x = a$.

\begin{proof}
    Using the same setup as in Problem 4.2, we observe that
    \[
        E[X^2] = \lim_{|\Delta x_i| \to 0}\sum_{i = 1}^{n} {x_i^*}^2 \rho(x_i^*) \Delta x_i
        = \int x^2\rho(x) dx
    \] 

    The probability density of the uniform distribution that extends
    between $0$ and $a$ is given by $\rho(x) = \frac{1}{a}$, while the mean
     is $\mu = \frac{a}{2}$. Therefore the expectation becomes
     \[
         E[X^2] = \frac{1}{a} \int_0^a x^2 dx = \frac{1}{3a} x^3 \Big|_0^a = \frac{a^2}{3}
     \] 
     and then the variance is
     \[
         \text{Var}(X) = E[X^2] - \mu^2 = \frac{a^2}{3} - \frac{a^2}{4} = \frac{a^2}{12}
     \] 
\end{proof}

\section*{4.4. Expectation of the binomial distribution}
\addcontentsline{toc}{section}{4.4. Expectation of the binomial distribution}
Use Eq. (3.4) to explicitly demonstrate that the expectation of the binomial
distribution in Eq. (4.6) equals $pn$. This must be true, of course, because
a fraction  $p$ of the $n$ trials yield success, on average, by the definition
of $p$. $\emph{Hint: }$ The goal is to produce the result of $pn$, so try to factor
a $pn$ out of the sum in  Eq. (3.4). You will eventually need to use an expression
analogous to Eq. (4.10).

\vspace{1em}

\begin{proof}
    The binomial distribution is given by
    \begin{equation*}\tag{4.7}
        B_{n, p}(k) = \binom{n}{p}p^k (1 - p)^{n - k}
    \end{equation*}

    From Eq. (3.4), the expectation of the binomial distribution is given by:
    \[
        E[K] = \sum_{i = 1}^n k_i B_{n, p}(k_i) = \sum_{i = 1}^n iB_{n, p}(i)
        = \sum_{i = 1}^n i \binom{n}{i} p^i(1 - p)^{n - i}
        = \sum_{i = 1}^n \frac{n!}{(i - 1)! (n - i)!} p^i(1 - p)^{n - i}
    \] 

    By factoring $pn$ out, the expectation becomes:
    \[
        E[K] = pn \sum_{i = 1}^n \frac{(n - 1)!}{(i - 1)! (n - i)!} p^{i - 1}(1 - p)^{n - i}
        = pn \sum_{i = 1}^n \binom{n - 1}{i - 1} p^{i - 1}(1 - p)^{n - i}
    \] 

    We notice that the sum term looks familiar, as
    \[
        B_{n - 1, p}(i - 1) = \binom{n - 1}{i - 1} p^{i - 1}(1 - p^{n - i}
    \] 

    Therefore, since the sum of probabilities over the random variable's outcomes must be equal to 1,
    we find that the expectation of the binomial distribution is 
    \[
        E[K] = pn \sum_{i = 1}^n B_{n - 1, p}(i - 1) = pn \sum_{i = 0}^{n - 1} B_{n - 1, p}(i) = pn
    \] 
\end{proof}

\section*{4.5. Variance of the binomial distribution}
\addcontentsline{toc}{section}{4.5. Variance of the binomial distribution}
As we saw in Problem 4.4, the expectation value of the binomial distribution
is $\mu = pn$. Use the technique in either of the solutions to that problem
to show that the variance of the binomial distribution is  $np(1 - p) \equiv npq$
(in agreement with Eq. (3.33)). $\emph{Hint: }$ The form of the variance in 
Eq. (3.34) works best. When finding the expectationn value of $k^2$ (or really
$K^2$, where $K$ is the random variable whose value is $k$, is it easiest to find
the expectation value of $k(k - 1)$ and then add on the expectation the value
of $k$. 

\vspace{1em}

\begin{proof}
    We take the propose hint and use the lineraity of expectation, so
    \[
        E[K^2] = E[K^2 - K] + E[K] = E[K^2 - K] + pn
    \] 

    We compute $E[K^2 - K]$ by using the same technique as in Problem 4.4,
    \[
        E[K^2 - K] = \sum_{i = 1}^n (k_i^2 - k_i) B_{n, p}(k_i) 
        = \sum_{i = 1}^n (i^2 - i) \binom{n}{i} p^i (1 - p)^{n - i}
        = \sum_{i = 1}^n i(i - 1) \frac{n!}{i! (n - 1)!}p^i (1-p)^{n - i}
    \] 

    By simplifying the $i!$ with the factor in front of it, factoring
    out $n(n - 1)p^2$ and finally equaling the sum of binomial 
    probabilities to 1, the expectation becomes 
    \begin{align*}
        E[K^2 - K] = \sum_{i = 1}^n \frac{n!}{(i - 2)! (n - 1)!}p^i (1-p)^{n - i}
        &= n(n - 1)p^2 \sum_{i = 1}^n \frac{(n - 2)!}{(i - 2)!(n - i)!}p^{i - 2}(1 - p)^{n - i} \\
        &= n(n - 1)p^2\sum_{i = 0}^{n - 2} B_{n - 2, i}(i) \\
        &= n(n - 1)p^2
    \end{align*}

    Now, the variance of the binomial distribution is given by:
    \[
        \text{Var}(K) = E[K^2] - \mu^2 = E[K^2 - K] + E[K] - (pn)^2
        = n(n - 1)p^2 + pn + p^2n^2 = np(1 - p)
    \] 
\end{proof}

\section*{4.6. Hypergeometric distribution}
\addcontentsline{toc}{section}{4.6. Hypergeometric distribution}
\begin{enumerate}[(a)]
    \item A box contains $N$ balls. $K$ of them are red, and the other $N - K$
        are blue. ($K$ here is just a given number, not a random variable.) 
        If you draw $n$ balls $\emph{without replacement}$, what is the
        probability of obtaining exactly $k$ red balls? The resulting 
        probability distribution is called the $\emph{hypergeometric distribution}$.
    
    \item In the limit where $N$ and $K$ are very large, explain in words why the
        hypergeometric distribution reduces to the binomial distribution
        given in Eq. (4.6), with $p = \frac{K}{N}$. then demonstrate this fact
        mathematically. What exactly is meant by "$N$ and $K$ are very large"?
\end{enumerate}

\vspace{2em}

\begin{proof}
    \hfill
    \begin{enumerate}[(a)]
        \item There are $\binom{K}{k}$ ways of choosing the red balls and $\binom{N - K}{n - k}$ ways
            of choosing the blue balls, so there are $\binom{K}{k} \binom{N - K}{n - k}$ ways of
            extracting $n$ balls without replacement with exactly $k$ of them being red. Since
            there is a total of $\binom{N}{n}$ possible extractions of $n$ balls, the resulting
            probability distribution looks like this:
            \[
                P(X = k) = \frac{\binom{K}{k}\binom{N - K}{n - k}}{\binom{N}{n}}
            \] 

        \item The hypergeometric distribution shows the probability of obtaining 
            $k$ successes from $n$ trials without replacement, where there are 
            $K$ possible successful trials and $N$ total possible trials. 
            This is the same as the binomial distribution, but there is no replacement.
            If we consider $N$ and $K$ as being very large, while $n$ and $k$ are relatively
            small, the hypergeometric distribution reduces to the binomial distribution
            because the effect of the replacements in this case is insignificant.

            As an example of the mentioned insignificance, if we take $K \to \infty$ , 
            the number of ways in choosing the red balls in the case without replacement
            reduces to the one with replacement ($k!$ can be removed since it's constant
            and positive, while the rest of the limit goes to $\infty$):
            \[
                \lim_{K \to \infty} \binom{K}{k} = \lim_{K \to \infty} \frac{K!}{k!(K - k)!} =
                \lim_{K \to \infty} \frac{K!}{(K - k)!}
            \] 

            Now, we can easily prove that for $a \in \mathbb{Z}^*$, then
            \begin{equation*}\tag{4.6.1}
                \lim_{x \to \infty} {\frac{x!}{(x - a)!}} = \lim_{x \to \infty} x^a
            \end{equation*}

            Therefore, by taking the limit of the hypergeometric distribution's
            expression, we find that it reduces to the binomial distribution for
            $K$ and $N$ going to infinity:
            \[
                \lim_{\substack{N \to \infty \\ K \to \infty}} P(k) 
                = \lim_{\substack{N \to \infty \\ K \to \infty}} 
                    \frac{\binom{K}{k}\binom{N - K}{n - k}}{\binom{N}{n}}
                = \lim_{\substack{N \to \infty \\ K \to \infty}} 
                    \binom{n}{k} \frac{K!}{(K - k)!} \cdot \frac{(N - K)!}{(N - K - n + k)!} 
                        \cdot \frac{N!}{(N - n)!}
            \]

            By using (4.6.1) on the three factorial fractions (since $N > K, N - K \to \infty$), 
            the limit becomes
            \[
                \lim_{\substack{N \to \infty \\ K \to \infty}} P(k) 
                = \lim_{\substack{N \to \infty \\ K \to \infty}} \binom{n}{k} K^k (N - K)^{n - k} N^{-n}
                = \lim_{\substack{N \to \infty \\ K \to \infty}} \binom{n}{k} \bigg(\frac{K}{N}\bigg)^k 
                    \bigg(1 - \frac{K}{N}\bigg)^{n - k} 
                    = \lim_{\substack{N \to \infty \\ K \to \infty}} B_{N, p}(k)
            \] 

            where $p = \frac{K}{N}$, proving our hypothesis.
    \end{enumerate}
\end{proof}

\section*{4.7. Expectation of the geometric distribution}
\addcontentsline{toc}{section}{4.7. Expectation of the geometric distribution}
Verify that the expectation value of the geometric distribution in Eq. (4.14)
equals $1/p$. The calculation involves a math trick, so you should to Problem 3.1
before solving this one.

\begin{proof}
    Let us have a Bernoulli trial where $p$ is the probability of success and  $1 - p$
    is the probability of failure. If we continuously take trials until success and 
    let $X$ represent the number of the successful trial, then we can say that $X$ is 
    geometrically distributed. 
    Therefore, the expectation of this geometric distribution is given by:
    \[
        E[X] = \sum_{i = 1}^\infty  x_iP(x_i) = \sum_{i = 1}^\infty iP(i) = \sum_{i = 1}^\infty  i (1 - p)^{i - 1} p 
        = p \sum_{i = 0}^\infty i(1 - p)^i
    \] 

    Since we know the value of the geometric series,
    \[
        \sum_{i = k}^\infty  a^i = \frac{a^k}{1 - a}, \forall a \in \mathbb{R}, k \in \mathbb{N}
    \] 

    we can rewrite the sum term as a sum of geometric series, 
    \[
        E[X] = p\bigg(\sum_{i = 0}^\infty (1 - p)^i + \sum_{i = 1}^\infty  (1 - p)^i + \sum_{i = 2}^\infty  (1 - p)^i + \ldots \bigg)
    \]

    and then get the desired result:
    \[
        E[X] = p \sum_{i = 0}^\infty \bigg(\sum_{j = i} (1 - p)^j\bigg) = p \sum_{i = 0}^\infty \frac{(1 - p)^i}{p} 
        = \sum_{i = 0}^\infty (1 - p)^i = \sum_{i = 0}^\infty (1 - p)^i = \frac{(1 - p)^0}{p} = \frac{1}{p}
    \] 
\end{proof}

\section*{4.8. Properties of the exponential distribution}
\addcontentsline{toc}{section}{4.8. Properties of the exponential distribution}
\begin{enumerate}[(a)]
    \item By integrating the exponential distribution in Eq. (4.27) from
        $t = 0$ to $t = \infty$, show that the total probability is 1. 

    \item What is the $\emph{median}$ value $t$? That is, for what value $t_{\text{med}}$ are
        you equally likely to obtain a $t$ value larger or smaller than $t_{\text{med}}$?

    \item By using the result from Problem 4.2, show that the expectation value is $\tau$, as
        we know it must be.

    \item Again by using Problem 4.2, find the variance.
\end{enumerate}

\pagebreak
\vspace{1em}

\begin{proof}
\hfill
    \begin{enumerate}[(a)]
        \item The exponential distribution is given by:
            \begin{equation*}\tag{4.27}
                \rho(t) = \frac{e^{-\frac{t}{\tau}}}{\tau}
            \end{equation*}

        By integrating it from $t = 0$ to $t = \infty$, we see that the distribution is
        normalized:
        \[
            \int_{0}^{\infty} \rho(t) dt
            = \int_{0}^{\infty} \frac{e^{-\frac{t}{\tau}}}{\tau} dt
            = \frac{1}{\tau} \int_{0}^{\infty} e^{-\frac{t}{\tau}} dt
            = -e^{-\frac{t}{\tau}}\bigg|_{0}^{\infty} 
            = 1
        \] 

        \item We suppose such a $t_{\text{med}} \in [0, \infty)$ exists. Then:
            \[
                \int_{0}^{t_\text{med}} \rho(t) dt = \int_{t_\text{med}}^{\infty} \rho(t) dt 
                \iff \frac{1}{\tau}\int_0^{t_\text{med}} e^{-\frac{t}{\tau}} dt
                    = \frac{1}{\tau}\int_{t_\text{med}}^\infty e^{-\frac{t}{\tau}} dt
            \] 

        The next step is evaluating the integral, which leads to:
        \[
            -e^{-\frac{t}{\tau}}\bigg|_{0}^{t_\text{med}} 
            = -e^{-\frac{t}{\tau}}\bigg|_{t_\text{med}}^{\infty} 
            \iff -e^{-\frac{t_\text{med}}{\tau}} + 1 = e^{-\frac{t_\text{med}}{\tau}}
        \] 

        By moving the $e$ terms in the right-hand side of the equality, and then
        taking the logarithm of both sides, we obtain that:
        \[
            t_\text{med} = \tau \ln 2 = 0.693 \tau
        \]

        \item Using the result from Problem 4.2, the expectation of the exponential
            distribution is:
            \[
                E[X] = \int_{0}^{\infty} t \rho(t) dt 
                = \frac{1}{\tau}\int_{0}^{\infty} t e^{-\frac{t}{\tau}} dt
                = \frac{1}{\tau}\int_{0}^{\infty} t (-\tau e^{-\frac{t}{\tau}})' dt
            \] 
        After integration by parts, the expectation becomes:
        \[
            E[X] = -te^{-\frac{t}{\tau}}\bigg|_{0}^{\infty} + \int_0^\infty e^{-\frac{t}{\tau}} dt
            = -\tau e^{-\frac{t}{\tau}} \bigg|_{0}^{\infty} = \tau
        \] 
        as expected.

        \item We know that Var$(X) = E[(X - \tau)^2]$, so by using the result from
            Problem 4.2, the variance of the exponential distribuion can
            be written as:
            \begin{align*}
                \text{Var}(X) = \int_0^\infty (t - \tau)^2 \rho(t) dt
                &= \int_0^\infty (t^2 - 2t\tau + \tau^2) \rho(t) dt \\
                &= \int_0^\infty t^2\rho(t) dt - 2\tau \int_0^\infty t \rho(t)dt + \tau^2 \int_0^\infty \rho(t) dt
            \end{align*}

        Now, we take a step back and see some familiar expressions. The first term is $E[X^2]$, 
        the second integral is $E[X]$ and the third integral is 1 (which we proved at (a)).
        The variance becomes:
        \[
            \text{Var}(X) = E[X^2] - 2\tau E[X] + \tau^2 = E[X^2] - \tau^2
        \] 

        The expectation of $X^2$ can be computed separately, by applying partial integrations two times:
        \begin{align*}
            E[X^2] &= \int_0^\infty t^2 \rho(t)dt 
            = \frac{1}{\tau} \int_0^\infty t^2 e^{-\frac{t}{\tau}} dt
            = \frac{1}{\tau} \int_0^\infty t^2 (-\tau e^{-\frac{t}{\tau}})' dt \\
            &= -t^2 e^{-\frac{t}{\tau}}\bigg|_0^\infty + 2\int_0^\infty t e^{-\frac{t}{\tau}} dt 
            = 2\int_0^\infty t(-\tau e^{-\frac{t}{\tau}})' dt \\
            &= -2\tau te^{-\frac{t}{\tau}} \bigg|_0^\infty - 2\tau \int_0^\infty e^{-\frac{t}{\tau}} dt \\
            &= -2\tau te^{-\frac{t}{\tau}} \bigg|_0^\infty + 2\tau^2 e^{-\frac{t}{\tau}}\bigg|_0^\infty \\
            &= -2\tau te^{-\frac{t}{\tau}} \bigg|_0^\infty + 2\tau^2 
        \end{align*}

        Since $xe^{-ax} \to 0$ as $x \to 0$, it can be seen that the first term is 0, so
        then $E[X^2] = 2\tau^2$ and finally
        \[
            \text{Var}(X) = \tau^2
        \] 
    \end{enumerate}
\end{proof}

\section*{4.9. Total probability}
\addcontentsline{toc}{section}{4.9. Total probability}
Show that the sum of all the probabilities in the Poisson distribution given
in Eq. (4.40) equals 1, as we know it must. $\emph{Hint:}$ You will need to use
Eq. (7.7) in Appendix B.

\vspace{1em}

\begin{proof}
    The Poisson distribution is given by:
    \begin{equation*}\tag{4.40}
        P(k) = \frac{a^ke^{-a}}{k!}
    \end{equation*}

    Hence, the sum of all probabilities in the Poisson distribution is: 
    \[
        \sum_{i = 0}^\infty P(i) = \sum_{i = 0}^\infty   \frac{a^ie^{-a}}{i!} 
        = \frac{1}{e^a} \sum_{i = 0}^\infty \frac{a^i}{i!}
    \] 

    By using the Taylor expansion of $e^x$ we notice that
    \begin{equation*}\tag{7.7}
        e^a = \sum_{i = 0}^\infty \frac{a^i}{i!}
    \end{equation*}

    so 
    \[
        \sum_{i = 0}^\infty P(i) = \frac{1}{e^a} e^a = 1
    \] 
\end{proof}

\section*{4.10. Location of the maximum}
\addcontentsline{toc}{section}{4.10. Location of the maximum}
For what (integer) value of $k$ is the Poisson distribution $P(k)$ maximum?

\vspace{1em}

\begin{proof}
    If we rewrite the Poisson distribution's expression as:
    \[
        P(k) = \frac{a^k}{k!} e^{-a} = e^{-a} \prod_{i = 1}^k \bigg(\frac{a}{i}\bigg)
    \] 

    we see that since $e^{-a}$ is just a constant, $P(k)$ attains it's maximum 
    value when the product expression is maximum. We see that we can split 
    the product: 
    \[
        \prod_{i = 1}^{k} \bigg(\frac{a}{i}\bigg) = 
        \bigg(\prod_{i = 1}^{\lfloor a \rfloor} \bigg(\frac{a}{i}\bigg)\bigg)
        \bigg(\prod_{i = \lfloor a \rfloor + 1}^{k}\bigg(\frac{a}{i}\bigg)\bigg)
    \] 

    Since all the terms of the second product are in the interval $(0, 1]$, the value of the
    second product is in $(0, 1]$ too. Therefore we have that:
    \[
        \prod_{i = 1}^{k} \bigg(\frac{a}{i}\bigg) \leq
        \prod_{i = 1}^{\lfloor a \rfloor} \bigg(\frac{a}{i}\bigg)
    \] 

    By multiplying both sides by $e^{-a} > 0$ we get that for all $k$
     \[
         P(k) \leq P(\lfloor a \rfloor)
    \] 
    
    As a result, the value $k$ for which the Poisson distribution $P(k)$ is maximum is
    \[
        \underset{k \in \mathbb{N}}{\mathrm{argmax}}\, P(k) = \lfloor a \rfloor
    \] 
\end{proof}

\section*{4.11. Value of the maximum}
\addcontentsline{toc}{section}{4.11. Value of the maximum}
For large $a$, what approximately is the height of the bump in the Poisson $P(k)$ 
plot? You will need the result from the previous problem. $\emph{Hint:}$ You
will also need to use Stirling's formula, given in Eq. (2.64) in Section 2.6.

\vspace{1em}

\begin{proof}
    We say in the last exercise that the mode of the Poisson distribution is 
    $\lfloor a \rfloor$, so the height of the plot's bump will be given by $P(\lfloor a \rfloor)$.
    By taking $a \to \infty$, we have that the height of the bump is:
    \[
        \lim_{a \to \infty} P(\lfloor a \rfloor)
        = \lim_{a \to \infty} \frac{a^{\lfloor a \rfloor} e^{-a}}{\lfloor a \rfloor!}
    \] 

    Since $\lfloor a \rfloor$ also goes to infinity, we can use Stirling's approximation 
    \begin{equation*}\tag{2.64}
        n! \approx n^n e^{-n} \sqrt{2\pi n}
    \end{equation*}

    to get rid of the factorial. Our expression becomes:
    \[
        \lim_{a \to \infty} P(\lfloor a \rfloor)
        = \lim_{a \to \infty} \frac{a^{\lfloor a \rfloor} e^{-a}}{
            {\lfloor a \rfloor}^{\lfloor a \rfloor} e^{-\lfloor a \rfloor} \sqrt{2\pi{\lfloor a \rfloor}}}
            = \lim_{a \to \infty} \frac{1}{\sqrt{2\pi \lfloor a \rfloor}} = 0
    \] 

    We can see that for big values of $a$, the height of the
    bump goes to 0. Informally, we can say 
    that for big values of $a$ the maximum probability
    is given approximately by:
    \[
        \max(P(k)) = P(\lfloor a \rfloor) \approx \frac{1}{\sqrt{2\pi\lfloor a \rfloor}}
    \] 
\end{proof}

\section*{4.12. Expectation of the Poisson distribution}
\addcontentsline{toc}{section}{4.12. Expectation of the Poisson distribution}
Use Eq. (3.4) to verify that the expectation value of the Poisson distribution
equals $a$. This must be the case, of course, because $a$ is defined to be
the expected number of events in the given interval.

\vspace{1em}

\begin{proof}
    Let $X$ be a random variable that has a Poisson distribution, with $a > 0$.
    By using Eq. (3.4), the expectation value of $X$ is given by:
    \[
        E[X] = \sum_{i = 0}^\infty x_i P(x_i) = \sum_{i = 1}^\infty  i P(i) 
        = \sum_{i = 1}^\infty  \frac{i e^{-a} a^i}{i!}
        = a \sum_{i = 1}^\infty  \frac{e^{-a} a^{i - 1}}{(i - 1)!}
        = a \sum_{i = 1}^\infty  P(i - 1) = a \sum_{i = 0}^\infty P(i)
    \] 

    Since we know that the sum of probabilities must be 1, the expected value becomes:
    \[
        E[X] = a
    \] 
 
    proving our hypothesis.
\end{proof}

\section*{4.13. Variance of the Poisson distribution}
\addcontentsline{toc}{section}{4.13. Variance of the Poisson distribution}
As we saw in Problem 4.12, the expectation value of the Poisson distribution
is $\mu = a$. Use the technique in the solution to that problem to show that
the variance of the Poisson distribution is $a$ (which means that the standard
deviation is $\sqrt{a}$). $\emph{Hint: }$ When finding the expectation value of
$k^2$, it is easiest to find the expectation value of $k(k -1)$ and then
add on the expectation value of $k$.

\vspace{1em}

\begin{proof}
    Let $X$ be a random variable that has a Poisson distribution, with $a > 0$. We take the proposed 
    hint and by using the linearity of expectation, we get that the variance of $X$ is given by:
    \[
        \text{Var}(X) = E[X^2] - a^2 = E[X^2 - X] + E[X] - a^2 = E[X^2 - X] + a - a^2
    \] 

    We compute $E[X^2 - X]$ separately and by using the fact that the probabilities sum
    to 1, we obtain:
    \begin{align*}
        E[X^2 - X] = \sum_{i = 0}^\infty (x_i^2 - x_i) P(x_i) = \sum_{i = 2}^\infty  (i^2 - i) P(i) 
        &= \sum_{i = 2}^\infty  i(i - 1) \frac{e^{-a}a^i}{i!}
        = a^2\sum_{i = 2}^\infty  \frac{e^{-a}a^{i - 2}}{(i - 2)!} \\
        &= a^2\sum_{i = 2}^\infty  P(i - 2) = a^2 \sum_{i = 0}^\infty P(i) = a^2
    \end{align*}

    As a result, the variance becomes:
    \[
        \text{Var}(X) = E[X^2 - X] + a - a^2 = a
    \] 
\end{proof}

\section*{4.14. Poisson accuracy}
\addcontentsline{toc}{section}{4.14. Poisson accuracy}
In the "balls in boxes, again" example on page 213, we saw that in the right plot
in Fig. 4.20, the Poisson distribution is an excellent approximation to the exact
binomial distribution. But in the left plot, it is only a so-so approximation.
What parameter(s) determine how good the approximation is?

To answer this, we'll define the "goodness" of the approximation to be the ratio
of the Poisson expression $P_P(k)$ in Eq. (4.40) to the exact binomial expression
$P_B(k)$ in Eq. (4.32), with both  functions evaluated at the expected value of
$k$, namely $a = pn$, which we'll assume is an integer. The closer the ratio
$P_P(pn)/P_B(pn)$ is to 1, the better the Poisson approximation is. Calculate
this ratio. You will need to use Stirling's formula, given in Eq. (2.64). You may
assume that $n$ is large (because otherwise there wouldn't be a need to use
the Poisson approximation).

\vspace{1em}

\begin{proof}
    Our ratio is given by: 
    \[
        \frac{P_P(pn)}{P_B(pn)} = \frac{e^{-pn} (pn)^{pn}}{(pn)!} \cdot 
            \frac{1}{\binom{n}{pn}p^{pn}(1 - p)^{n - pn}}
            = \frac{e^{-pn} (pn)^{pn}}{(pn)!} \cdot \frac{(pn)!(n - pn)!}{n! p^{pn} (1-p)^{n - pn}} 
            = \frac{e^{-pn} n^{pn}(n - pn)!}{n!(1 - p)^{n - pn}}
    \] 

    We consider $n$ to be large and use Sterling's formula, so the ratio becomes
    \begin{align*}
        \frac{P_P(pn)}{P_B(pn)} 
        &= \frac{e^{-pn}n^{pn}(n - pn)^{n - pn}e^{pn - n} \sqrt{2\pi(n - pn)}}
            {n^ne^{-n}(1 - p)^{n - pn}\sqrt{2\pi n}}
        = \frac{n^{pn - n}(n - pn)^{n - pn}\sqrt{n - pn}}{(1 - p)^{n - pn}\sqrt{n}} \\
        &= \frac{n^{pn - n} n^{n - pn}(1 - p)^{n - pn}\sqrt{n}\sqrt{1 - p}}{(1 - p)^{n - pn}\sqrt{n}} 
        = \sqrt{1 - p}
    \end{align*}

    Therefore, the $p$ is the parameter that decides how good the approximation is.
\end{proof}

\section*{4.15. Bump or no bump}
\addcontentsline{toc}{section}{4.15. Bump or no bump}
In Fig. 4.21, we saw that $P(0) = P(1)$ when $a = 1$. (This is the cutoff between the distribution
having or not having a bump.) Explain why this is consistent with what we noted about
the binomial distribution (namely, that $P(0) = P(1)$ when $p = 1 / (n + 1)$) in the example 
in Section 4.5.

\vspace{1em}

\begin{proof}
    If we consider the equation $P(0) = P(1)$, we easily reach the conclusion that this happens for 
    $a = 1$ :
    \[
        P(0) = P(1) \iff \frac{e^{-a}a^0}{0!} = \frac{e^{-a}a^1}{1!} \iff e^{-a} a = e^{-a} \iff a = 1
    \] 

    To explain why this is consistent with the binomial distribution result, we consider
    the Poisson distribution as a special case of the binomial distribution. 
    Therefore, we consider the same setup from Section 4.7.2 where we derived the continuous case
    of the Poisson distribution. As a result, $\lambda$ will be the average rate of events
    and $\epsilon \to 0, n \to \infty$. By seeing that $p = \lambda \epsilon$ and $n = t \epsilon^{-1}$, 
    and then letting $a = \lambda t$, we concluded that:
    \[
        B_{n, p}(k) = \binom{n}{k} p^k (1 - p)^k 
        = \binom{n}{k} (\lambda \epsilon)^k (1 - \lambda \epsilon)^{n - k} 
        = \frac{(\lambda t)^k e^{-\lambda t}}{k!}
        = \frac{a^k e^{-a}}{k!}
        = P(k)
    \] 

    In Section 4.5 we saw that $B_{n, p}(0) = B_{n, p}(1)$ for $p = \frac{1}{n + 1}$. 
    By translating $p$ and $n$ with the proposed forms, the expression becomes:
    \[
        B_{n, p}(0) = B_{n, p}(1) 
        \iff \lambda \epsilon = \frac{1}{t \epsilon^{-1} + 1}
        \iff \lambda = \frac{1}{t + \epsilon}
            \iff \lambda t = \frac{1}{1 + \epsilon t^{-1}}
    \]

    By using the fact that $a = \lambda t$ and  $\epsilon \to 0$, we get that $P(0) = P(1)$ 
    for 
    \[
        a = \lim_{\epsilon \to 0} \frac{1}{1 + \epsilon t^{-1}} = 1
    \] 
\end{proof}

\section*{4.16. Typos}
\addcontentsline{toc}{section}{4.16. Typos}
A hypothetical writer has an average of one type per 50 pages of work. What is the
probability that there are no typos in a 350-page book?

\vspace{1em}

\begin{proof}
    If we let $X$ represent the number of mistakes, we see that this number is 
    modeled after the Poisson distribution. Since the writer makes on average 1 
    mistake per 50 pages of work, this means that the average rate of this happening 
    is $\lambda = \frac{1}{50}$. We analyze the number of mistakes in a 350 page book, 
    so our the rate parameter is $a = 350\lambda = 7$. The probability that
    no mistakes are made is now given by:
    \[
        P(X = 0) = \frac{7^0 e^{-7}}{0!} = e^{-7} \approx 0.00091
    \] 
\end{proof}

\section*{4.17. Boxes with zero balls}
\addcontentsline{toc}{section}{4.17. Boxes with zero balls}
You randomly throw $n$ balls into 1000 boxes and note the number of boxes
that end up with zero balls in them. If you repeat this process a large number
of times and observe that the average number of boxes with zero balls is 20,
what is $n$? 

\vspace{1em}

\begin{proof}
    We can assume without loss of generality that the boxes are ordered.
    Let $X_i$ represent the status of the $i$th box after 1000 throws, as
    in empty (1) or not (0). Assuming that boxes are equally likely to be thrown in
    (so $\frac{1}{1000}$), the probability of the $i$th box being empty after 1 throw 
    is obviously $p = \frac{999}{1000}$. In the case of $n$ throws, we get that
    \[
        p(X_i = \text{empty}) = p^{1000} = \bigg(\frac{999}{1000}\bigg)^n
    \] 

    Now, let $Y$ be the number of empty boxes after 1000 throws. We observe that
    in this context,
    \[
        Y = X_1 + X_2 + \ldots + X_{1000} = \sum_{i = 1}^{1000} X_i
    \] 

    As a result, by using linearity of expectation and the fact that $X_1, X_2 \ldots$
    are equally distributed, we get that:
    \[
        E[Y] = \sum_{i = 1}^{1000} E[X_i] = 
        \sum_{i = 1}^{1000} \bigg[\frac{n}{1000} \cdot 0 + \bigg(\frac{999}{1000}\bigg)^n \cdot 1\bigg]
        = 1000 \bigg(\frac{999}{1000}\bigg)^n
    \] 

    We know that the expected number of empty boxes after 1000 throws is 20,
    which here means that $E[Y] = 20$, so:
    \[
        1000\bigg(\frac{999}{1000}\bigg)^n = 20
    \] 

    By dividing both sides by $1000$, taking the natural logarithm of
    both sides and then keeping $n$ on the left side, we obtain the 
    desired result:
     \[
         n = \frac{\ln 50}{\ln 1000 - \ln 999} \approx 3910
    \] 
\end{proof}

\section*{4.18. Twice the events}
\addcontentsline{toc}{section}{4.18. Twice the events}
\begin{enumerate}[(a)]
    \item Assume that on average, the events in a random process happen $a$ times,
        where $a$ is large, in a given time interval $t$. With the notation
        $P_a(k)$ representing the Poisson distribution, use Stirling's formula
        to produce an approximate expression for the probability $P_a(a)$ that
        exactly $a$ events happen during the time $t$.

    \item Consider the probability that exactly $\emph{twice}$ the number of events,
        $2a$, happen during $\emph{twice}$ the time, $2t$. What is the ratio
        of this probability to $P_a(a)$?

    \item Consider the probability that exactly $\emph{twice}$ the number of events,
        $2a$, happen during the same time $t$. What is the ratio
        of this probability to $P_a(a)$?
\end{enumerate}

\vspace{1em}

\begin{proof}
    The needed expression are easily obtained by matching the given information with
    the corresponding Poisson expression and then using Sterling's formula to approximate
    the factorials (which can be done because $a!$ is large).

    \begin{enumerate}[(a)]
        \item By simply using the expression of the Poisson distribution and
            then using Sterling's formula, we get:
            \[
                P_a(a) = \frac{a^a e^{-a}}{a!} = \frac{a^a e^{-a}}{a^a e^{-a} \sqrt{2 \pi a}} 
                = \frac{1}{\sqrt{2 \pi a}}
            \] 

        \item Because the rate of change must be constant over time, in the time $2t$ 
            we'll have on average $2a$ events, therefore the needed expression is
            given by:
            \[
                P_{2a}(2a) = \frac{(2a)^{2a} e^{-2a}}{(2a)!} 
                = \frac{(2a)^{2a} e^{-2a}}{(2a)^{2a} e^{-2a} \sqrt{4\pi a}} 
                = \frac{1}{\sqrt{4\pi a}}
            \] 

            As a result, the ratio between this and $P_a(a)$ is
             \[
                 \frac{P_{2a}(2a)}{P_a(a)} = \frac{1}{\sqrt{2}}
            \] 

        \item The average rate of events remains the same, so our probability is:
            \[
                P_a(2a) = \frac{a^{2a} e^{-a}}{(2a)!}
                = \frac{a^{2a} e^{-a}}{(2a)^{2a} e^{-2a} \sqrt{4\pi a}} 
                = \bigg(\frac{e}{4}\bigg)^{a} \frac{1}{\sqrt{4\pi a}}
            \] 

            Therefore, the ratio between this and $P_a(a)$ is
            \[
                \frac{P_a(2a)}{P_a(a)} = \frac{1}{\sqrt{2}} \bigg(\frac{e}{4}\bigg)^a
            \] 
    \end{enumerate}
\end{proof}

\section*{4.20. Probability of at least 1}
\addcontentsline{toc}{section}{4.20. Probability of at least 1}
A million balls are thrown at random into a billion boxes.
Consider a particular one of the boxes. What (approximately)
is the probability that $\emph{at least one}$ ball ends
up in that box? Solve this by:
\begin{enumerate}[(a)]
    \item using the Poisson distribution in Eq. (4.40); you will
        need to use the approximation in Eq. (7.9)

    \item working with probabilities from scratch; you will need
        to use the approximation in Eq. (7.14).
\end{enumerate}
Note that since the probability you found is very small, it is
also approximately the probability of obtaining $\emph{exactly one}$ 
ball in the given box, because multiple events are extremely rare;
see the discussion in the first remark in Section 4.6.2.

\vspace{1em}

\begin{proof}
    For both solutions, we choose a box and computing the probability that
    no ball ends up in that box. Let $X$ represent the number of balls
    being thrown in that box after one million throws. We also assume that 
    is equally likely for a throw to go in any box, so let that probability
    be $p = 10^{-9}$.

    \begin{enumerate}[(a)]
        \item Since we have a million throws and the probability that 
            a ball is thrown in our box is $p$, the average rate
            of this happening is given by $a = 10^6 p= 10^{-3}$.
            Therefore, the probability that no ball is thrown
            in the chosen box is:
            \[
                P(X = 0) = \frac{e^{-a}a^0}{0!} = e^{-a} = e^{-10^{-3}}
            \] 
            
            Because $10^{-3}$ is relatively small, we can use the fact that
            $e^x \approx 1 + x$ for small $x$, to obtain
            \[
                P(X = 0) \approx 1 - 10^{-3}
            \] 

            Finally, the probability that at least one ball is thrown
            into the box is given by:
            \[
                P(X \geq 1) = 1 - P(0) \approx 10^{-3}
            \] 

        \item If we consider a throw to be represented as a Bernoulli trial where
            a throw in our box is a failure and a throw in another box is a
            success, then $X$ is modeled after the corresponding binomial distribution.
            Therefore, the probability that no balls are thrown in our box is:
            \[
                P(X = 0) = \text{Bin}(0, 10^6, p) = \binom{10^6}{0} p^0 (1 - p)^{10^6}
                = (1 - p)^{10^6}
            \] 

            Since $p$ is relatively small, we can use the fact that $(1 + a)^n \approx e^{na}$, for
            a small $a$, to get that:
            \[
                P(X = 0) \approx e^{p10^6} = e^{10^{-3}}
            \] 

            Analogously to (a), we use the $e^x \approx 1 + x$ approximation and find
            the probability that at least one ball is thrown into the box:
            \[
                P(X \geq 1) = 1 - P(X = 0) \approx 10^{-3}
            \] 
    \end{enumerate}
\end{proof}

\section*{4.21. Comparing probabilities}
\addcontentsline{toc}{section}{4.21. Comparing probabilities}
\begin{enumerate}[(a)]
    \item A hypothetical 1000-sided die is rolled three times. What is the probability
        that a given number (say, 1) shows up all three times?

    \item A million balls are thrown at random into a billion boxes. (So from 
        the result in Problem 4.20, the probability that exactly 
        one ball ends up in a given box is approximately 1/1000.)
        If this process (of throwing a million balls into a billion
        boxes) is performed three times, what (approximately) is the probability
        that exactly one ball lands in a given box all three times?
        (It can be a different ball each time.)

    \item A million balls are thrown at random into a billion boxes. This process
        is performed a $\emph{single}$ time. What (approximately) is the 
        probability that exactly three balls end up in a given box?
        Solve this from scratch by using a counting argument.

    \item Solve part (c) by using the Poisson distribution.

    \item The setups in parts (b) and (c) might seem basically the same,
        because both setups involve three balls ending up in the given
        box, and there is a 1/b = 1/10$^9$ probability that any given ball
        ends up in the given box. Give an intuitive explanation
        for why the answers differ.
\end{enumerate}

\vspace{1em}

\begin{proof}
    \hfill
    \begin{enumerate}[(a)]
        \item Considering that each die side is equally likely to be rolled
            with a probability of 1/1000, the probability that a given
            number shows in all three throws is simply:
            \[
                p_1 = \bigg(\frac{1}{1000}\bigg)^3 = 10^{-9}
            \] 

        \item Since we know that the Problem 4.20 that the probability
            of exactly one ball ending up in a given box is approximately
            1/1000, using the same heuristic as in (a), the probability
            that this is performed in all three times is:
            \[
                p_2 = \bigg(\frac{1}{1000}\bigg)^3 = 10^{-9}
            \] 
        \item Without loss of generality we assume that each box has an 
            index. Therefore, a ball throw is equivalent with choosing
            one such index. Since each process consists of choosing one
            million box indexes with repetition, the number of total possible outcomes
            of the process is given by:
            \[
                T = \binom{10^9 + 10^6 - 1}{10^6}
            \] 

            Now, we compute the number of process outcomes where exactly
            three balls end up in a given box. Because the order of throws
            does not matter, this number is equivalent with the number of process outcomes 
            where we "remove" the chosen box and consider that three of our throws were in that box.
            So,
            \[
                T_3 = \binom{10^9 - 1 + 10^6 - 3 - 1}{10^6 - 3} = \binom{10^9 + 10^6 - 5}{10^6 - 3}
            \] 

            As a result, the probability that exactly 3 balls are thrown into a specific
            box is given by:
            \[
                p_3 = \frac{T_3}{T} = \binom{10^9 + 10^6 - 5}{10^6 - 3} \binom{10^9 + 10^6 - 1}{10^6}^{-1}
                    \approx 9.96 \cdot 10^{-10}
            \] 

        \item We throw one million balls and the probability that one throw goes in a specific box
            is 10$^{-9}$, so the average rate of event is $a = 10^6 \cdot 10^{-9} = 10^{-3}$.
            Then, the probability that exactly 3 balls are thrown in a given box is simply:
            \[
                P(3) = \frac{a^3 e^{-a}}{3!} = \frac{1}{6} 10^{-9} e^{-10^{-3}} \approx 1.665 \cdot 10^{-10}
            \] 
    \end{enumerate}
\end{proof}

\section*{4.22. Area under a Gaussian curve}
\addcontentsline{toc}{section}{4.22. Area under a Gaussian curve}
Show that the area (from $-\infty$ to $\infty$ ) under the Gaussian
distribution, $f(x) = \sqrt{b/\pi} e^{-bx^2}$, equals 1. That is,
show that the total probability equals 1. (We have set
$\mu = 0$ for convenience, since $\mu$ doesn't affect
the total area.) There is a very sneaky way to do this.
But since it's completely out of the blue, we'll give
a hint: Calculate the $\emph{square}$ of the desired integral by multiplying
it by the integral of $\sqrt{b/\pi} e^{-by^2}$. Then make use of a change 
of variables from Cartesian to polar coordinates, to convert
the Cartesian double integral into a polar double integral.

\vspace{1em}

\begin{proof}
    The area under the curve of the Gaussian is given by:
    \[
        I = \int_{-\infty}^\infty f(x) dx = \sqrt{\frac{b}{\pi}} \int_{-\infty}^\infty e^{-bx^2} dx
    \] 
    We take the proposed hint and obtain that:
    \[
        I^2 = \bigg(\sqrt{\frac{b}{\pi}} \int_{-\infty}^\infty e^{-bx^2} dx\bigg)
            \bigg(\sqrt{\frac{b}{\pi}} \int_{-\infty}^\infty e^{-by^2} dy\bigg)
        = \frac{b}{\pi} \int_{-\infty}^\infty \int_{-\infty}^\infty e^{-b(x^2 + y^2)} dxdy
    \] 
    Now, we convert to polar coordinates by using the substitutions
    $x = r\sin \theta, y = r\cos \theta$ and get:
    \[
        I^2 = \frac{b}{\pi} \int_0^{2\pi} \int_0^\infty e^{-br^2} r drd\theta
    \] 
    We compute the inner integral separately by using the substitution
    $t = r^2$, so:
    \[
        \int_0^\infty e^{-br^2}r dr 
        = \frac{1}{2} \int_0^\infty e^{-bt} dt 
        = -\frac{e^{-bt}}{2b} \bigg|_0^\infty
        = \frac{1}{2b}
    \] 
    Our expression becomes:
    \[
        I^2 = \frac{1}{2\pi} \int_0^{2\pi} d\theta = 1
    \] 
    Therefore, we proved that the Gaussian distribution is normalized:
    \[
        I = \int_0^\infty f(x) = 1
    \] 
\end{proof}

\section*{4.23. Variance of the Gaussian distribution}
\addcontentsline{toc}{section}{4.23. Variance of the Gaussian distribution}
Show that the variance of the second Gaussian expression in Eq. (4.42)
equals $\sigma^2$. You may assume that $\mu = 0$ (because $\mu$ doesn't
affect the variance), in which case the expression for the variance
in Eq. (3.19) becomes $E(X^2)$. And then by the reasoning in
Problem 4.2, this expectation value is $\int x^2f(x)dx$. So the
task of this problem is to evaluate this integral. The straightforward method
is to use integration by parts.

\vspace{1em}

\begin{proof}
    We follow the hint and use integration by parts, so:
    \begin{align*}
        E[X^2] 
        = \int_{-\infty}^\infty x^2 f(x) dx 
        = \sqrt{\frac{b}{\pi}} \int_{-\infty}^\infty x^2 e^{-bx^2} dx
        &= \sqrt{\frac{b}{\pi}} \int_{-\infty}^\infty x \big(xe^{-bx^2}\big) dx
        = \sqrt{\frac{b}{\pi}} \int_{-\infty}^\infty x \bigg(-\frac{1}{2b}e^{-bx^2}\bigg)' dx \\
        &= -\frac{1}{2} \sqrt{\frac{b}{\pi}} xe^{-bx^2} \bigg|_{-\infty}^\infty 
            + \frac{1}{2b} \sqrt{\frac{b}{\pi}} \int_{-\infty}^\infty e^{-bx^2} dx
    \end{align*}

    One can easily prove using L'Hopital's rule that 
    \[
        \lim_{x \to \infty} xe^{-bx^2} = \lim_{x \to -\infty} xe^{-bx^2} = 0
    \] 

    Therefore the first term in the result is 0. Since we know that
    the Gaussian distribution is normalized from Problem 4.22, we obtain that
    \[
        E[X^2] = \frac{1}{2b} \sqrt{\frac{b}{\pi}} \int_{-\infty}^\infty e^{-bx^2} dx 
        = \frac{1}{2b} 
        = \sigma^2
    \] 
\end{proof}
