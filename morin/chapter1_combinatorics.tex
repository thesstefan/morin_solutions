\chapter{Combinatorics}

\section*{1.14. Yahtzee}
In the game of Yahtzee, five dice are rolled in a group, with the order not mattering.

\begin{enumerate}[(a)]
    \item Using Eq. (1.16), how many unordered rolls (sets) are possible?

    \item In the spirit of the examples at the beginning of Section 1.7. reproduce the
        result in part (a) by determining how many unordered rolls there are
        of each general type (for example, three of one number and two of another, etc.)

    \item In the spirit of the example at the end of Section 1.7., show that the total
        number of \emph{ordered} Yahtzee rolls is $6^5 = 7776$.
\end{enumerate}

\vspace{2em}

\begin{proof}
    \hfill
    \begin{enumerate}[(a)]
        \item The number of unordered rolls is given by 
            \[
                \binom{5 + (6 - 1)}{6 - 1} = \binom{10}{5} = 252
            .\] 

        \item We split the rolls in 7 types:
            \begin{enumerate}[(1)]
                \item All five rolls are the same (e.g. 66666). There are 6 sets of this type.

                \item Four rolls have the same value and the other roll has another value (e.g. 66665).
                    There are $6 \cdot 5 = 30$ sets of this type.
                
                \item Three rolls have the same value and the other two rolls have other (different 2 by 2)
                    values (e.g. 66654). There are  $6 \binom{5}{2} = 6 \cdot 10 = 60$ sets of this type.

                \item Three rolls have the same value and the other two rolls have both another (same) value 
                    (e.g. 66655). There are $6 \cdot 5 = 30$ sets of this type.

                \item Two rolls have the same value and the other three rolls have different (2 by 2) values
                    (e.g. 66654). There are $6 \binom{5}{2} = 60$ such sets.

                \item Two rolls have the same value, two rolls have another (same value) and the 
                    remaining roll has another different value (e.g. 66554). 
                    There are $6 \binom{5}{2} = 60$ such sets of rolls.

                \item Each roll has a different value. (e.g. 65432). There are $\binom{6}{5} = 6$ such sets.
            \end{enumerate}

        By summing all the types, we get that the number of unordered rolls is 
        \[
            6 + 30 + 60 + 30 + 60 + 60 + 6 = 252
        .\] 

    \item We want to see in how many ways can the 5 roll types be ordered:
        \begin{enumerate}[(1)]
            \item All rolls are the same, so the set can be ordered in one way. The number of 
                ordered sets with the same 5 rolls is 6.

            \item Four rolls have the same value and the other roll has another value.
                The set can be ordered in 5 ways (we consider the different roll on each
                possible position). Therefore, there are $5 \cdot 30 = 150$ ordered sets of this type.

            \item Three rolls have the same value and the other two have other (different 2 by 2) values.
                The positions of the repeated values can be chosen in $\binom{5}{3} = 10$ and then there 
                are two ways to order the other values, so there are $10 \cdot 2 = 20$ ways to order the set.
                As a result, there are $20 \cdot 60 = 1200$ ordered sets of this type.

            \item Three rolls have the same value and the other two rolls have both another (same) value.
                There are $\binom{5}{3} = 10$ ways the sets can be ordered, so we get $10 \cdot 30 = 300$
                such ordered sets.

            \item Two rolls have the same value and the other three have other (different 2 by 2) values.
                As before, we find that the set of values can be ordered in 
                $\binom{5}{2} \cdot 3 \cdot 2 = 60$ ways.
                Hence, there are $60 \cdot 60 = 3600$ such ordered sets.

            \item Two rolls have the same value, two rolls have another (same value) and the 
                remining roll has another differente value. There are 
                $\binom{5}{2}\binom{3}{2} = 10 \cdot 3 = 30$ ways to order this set, so we get 
                $30 \cdot 60 = 1800$ such ordered sets.

            \item All rolls have different values. The sets can be ordered in $5! = 120$ ways. As a result,
                there are $120 \cdot 6 = 720$ such ordered sets.
        \end{enumerate}

        In conclusion, we get that the number of $\emph{ordered}$ sets of Yahtzee rolls is:
         \[
            6 + 150 + 1200 + 300 + 3600 + 1800 + 720 = 7776 = 6^5
        .\] 
    \end{enumerate}
\end{proof}

\section*{1.16. Pascal sum 2}
At the end of Section 1.8.3, we demonstrated the relation 
$\binom{n}{k} = \binom{n - 1}{k - 1} + \binom{n - 1}{k}$
by using the argument involving committees. Repeat this reasoning,
but now in terms of:
\begin{enumerate}[(a)]
    \item coin flips,
    \item the $(a + b)^n$ binom expansion.
\end{enumerate}

\vspace{2em}

\begin{proof}
    \hfill
    \begin{enumerate}[(a)]
        \item The number of ways we can get $k$ tails from $n$ coin flips is given by 
            $\binom{n}{k}$. If we single out the first flip, we get two cases:
            \begin{enumerate}[(1)]
                \item The flip was heads, so the number of ways to get tails $k$ times
                    from the rest $n - 1$ of the flips is $\binom{n - 1}{k}$

                \item The flip was tails, so the number of ways to get the remaining 
                    $k - 1$ tails flips from the other $n - 1$ rolls is given by $\binom{n - 1}{k - 1}$
            \end{enumerate}

        Therefore, the number of ways we get $k$ tails from $n$ coin flips can be split into the above
        two cases, so:
        \[
            \binom{n}{k} = \binom{n - 1}{k} + \binom{n - 1}{k - 1}
        .\] 

    \item The coefficient of the term $a^{n - k}b^k$ from the expansion of $(a + b)^n$ is given by 
        $\binom{n}{k}$. If we single out the first $(a + b)$ factor, we have two possible situations:
        \begin{enumerate}
            \item The factor was used in getting a power of $b$ in $a^{n - k}b^k$, so there are 
                $\binom{n - 1}{k - 1}$ factors to use for the other $k - 1$ powers of $b$ and the $n - k$ 
                powers of $a$, since 
                $\binom{n - 1}{n - k} = \binom{n - 1}{k - 1}$.

            \item The factor wasn't used in getting a power of $b$ in $a^{n - k}b^k$, so there are 
                $\binom{n - 1}{k}$ factors to use for the other $k$ b powers and the other $n - k - 1$ 
                powers of a, since $\binom{n - 1}{n - k - 1} = \binom{n - 1}{k}$.
        \end{enumerate}

    Therefore, the coefficient of $a^{n-k}b^k$ is given by:
    \[
        \binom{n}{k} = \binom{n - 1}{k} + \binom{n - 1}{k - 1}
    .\] 
    \end{enumerate}
\end{proof}

\section*{1.17. Pascal diagonal sum}
\begin{enumerate}[(a)]
    \item If we pick an unordered committee of three people from five people (A, B, C, D, E),
        we can list the $\binom{5}{3} = 10$ possibilities as shown in Table 1.19.
        We have grouped them according to which letter comes first. (The order of letters
        doesn't matter, so we've written each triplet in increasing alphabetical order.)
        The columns in the table tell us that we can think of $10$ as equaling $6 + 3+ 1$.
        Explain why it makes sense to write this sum as  $\binom{4}{2} + \binom{3}{2} + \binom{2}{2}$.

        \begin{table}[h]
            \centering
            \begin{tabular}{ccc}
                A B C & & \\ 
                A B D & & \\
                A B E & & \\
                A C D & B C D & \\
                A C E & B C E & \\
                A D E & B D E & C D E \\
            \end{tabular}
            \caption*{\textbf{Table 1.19:} Unordered triplets chosen from five people.}
        \end{table}

    \item You can also see from Table 1.15 and 1.16 that, for example 
        $\binom{6}{3} = \binom{5}{2} + \binom{4}{2} + \binom{3}{2} + \binom{2}{2}$.
    More generally,
    \begin{equation*}\tag{1.29}
        \binom{n}{k} = \binom{n - 1}{k - 1} + \binom{n - 2}{k - 2} + \binom{n - 3}{k - 3} + \ldots + 
                        \binom{k}{k - 1} + \binom{k - 1}{k - 1}.
    \end{equation*}

    In words: A given number (for example, $\binom{6}{3}$) in Pascal's triangle equals the sum
    of the numbers in the diagonal string that starts with the number that is above and to the 
    left of the given number ($\binom{5}{2}$ in this case) and then proceeds upward to the right.
    So the string contains $\binom{5}{2}, \binom{4}{2}, \binom{3}{2}$ and $\binom{2}{2}$
    in this case.

    Prove Eq.(1.29) by making repeated use of Eq.(1.22), which says that each number in Pascal's
    triangle is the sum of the two numbers abot it (or just the "1" above it, if it occurs at the end 
    of a line).
\end{enumerate}

\begin{proof}
    \hfill
    \begin{enumerate}[(a)]
        \item We split the committees into 4 categories:
            \begin{enumerate}[(1)]
                \item The set contains A, so there are $\binom{4}{2} = 6$ such sets.
                \item The set contains B and doesn't contain A, so there are $\binom{3}{2} = 3$ sets.
                \item The set contains C and doesn't contain A or B, so there is $\binom{2}{2} = 1$
                \item The set doesn't contain A, B or C. There are obviously no such sets, as we
                    need 3 elements.
           \end{enumerate}

        Since the reunion of those sets contains all the possible unordered committees of 3 members,
        we obtain that:
        \[
            \binom{4}{2} + \binom{3}{2} + \binom{2}{2} = 6 + 3 + 1 = 10 = \binom{5}{3}
        .\] 
        
    \item We prove using induction that:
    \begin{equation*}\tag{1.29}
        \binom{n}{k} = \binom{n - 1}{k - 1} + \binom{n - 2}{k - 2} + \ldots + 
                        \binom{k}{k - 1} + \binom{k - 1}{k - 1}
                        = \sum_{i = k}^{n} \binom{i - 1}{k - 1}, \forall k \in \mathbb{N^*}, k \leq n
    \end{equation*}
    for all $n \in \mathbb{N}, n \geq 2$.

    The base case is obviously valid, since
    \begin{align*}
        \binom{2}{1} &= \binom{1}{0} + \binom{0}{0} = 1 + 1 = 2, \\
        \binom{2}{2} &= \binom{1}{1} + \binom{0}{1} = 1 + 0 = 1
    .\end{align*}

    We assume that the relation holds for a fixed $m \in \mathbb{N}, m \geq 2$, so we have:
    \[
        \binom{m}{k} = \sum_{i = k}^{m}\binom{i - 1}{k - 1}, \forall k \in \mathbb{N^*}, k \leq n
    .\]

    Using (1.22), we obtain that: 
    \[
        \binom{m + 1}{k} = \binom{m}{k} + \binom{m}{k - 1} 
                         = \sum_{i = k}^{m}\binom{i - 1}{k - 1} + \binom{m}{k - 1} 
                         = \sum_{i = k}^{m + 1}\binom{i - 1}{k - 1} 
    .\] 

    Since (1.22) holds true for the base case and the $m$ case implies the validity of the $m + 1$ case,
    we proved using induction that (1.29) holds for all $n \in \mathbb{N}, n \geq 2$.
\end{enumerate}
\end{proof}
